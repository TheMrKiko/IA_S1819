\documentclass[12pt,a4paper]{article}
\usepackage[utf8]{inputenc}
\usepackage[portuguese]{babel}
\usepackage{graphicx, hyperref, verbatim, multicol}

\author{Grupo 74 \and Daniel Fernandes 86400 \& Francisco Sousa 86416}
\title{Relatório 1ª Parte Projeto de IA}
\begin{document}
\maketitle

\abstract{Nesta primeira parte do projeto, foi-nos pedido um programa em Python que resolva
	diferentes puzzles de uma variante do jogo Solitaire. O objectivo é encontrar uma sequência de jogadas,
	independentemente do tamanho do tabuleiro e da sua configuração inicial.}

Irá ser realizada uma análise para os seguintes problemas dados:

\begin{multicols}{2}
	\paragraph{Problema 1 (5x5)}
	\begin{verbatim}
		[["_","O","O","O","_"],
		["O","_","O","_","O"],
		["_","O","_","O","_"],
		["O","_","O","_","_"],
		["_","O","_","_","_"]]
	\end{verbatim}
	\paragraph{Problema 2 (4x4)}
	\begin{verbatim}
		[["O","O","O","X"],
		["O","O","O","O"],
		["O","_","O","O"],
		["O","O","O","O"]]
	\end{verbatim}
	\paragraph{Problema 3 (4x5)}
	\begin{verbatim}
		[["O","O","O","X","X"],
		["O","O","O","O","O"],
		["O","_","O","_","O"],
		["O","O","O","O","O"]]
	\end{verbatim}
	\paragraph{Problema 4 (4x6)}
	\begin{verbatim}
	[["O","O","O","X","X","X"],
	["O","_","O","O","O","O"],
	["O","O","O","O","O","O"],
	["O","O","O","O","O","O"]]
	\end{verbatim}
\end{multicols}


\section{Tempo de Execução}

A tabela \ref{te} apresenta os dados para este teste.

\begin{table}[ht]
	\centering
	\begin{tabular}{c|c|c|c|c}
		\multicolumn{1}{r|}{} & \textbf{Problema 1} & \textbf{Problema 2} & \textbf{Problema 3} & \textbf{Problema 4} \\ \hline
		\textbf{Greedy} & 0.005665  & 0.021978 & 1.768940  & 5.182601 \\ \hline
		\textbf{A*} & 0.009196 & 0.124620  & 1.741809 & 1.480939 \\\hline
		\textbf{DFS}  & 0.004827 & 0.811957  & 8.112057  & -
	\end{tabular}
	\caption{Tempo de execução para os quatro problemas}
	\label{te}
\end{table}

A DFS faz o tabuleiro menor mais rapidamente porque os outros
gastam muito tempo a expandir nós desnecessários. No entanto,
não consegue resolver o último problema em tempo útil porque é
muito complexo.

\section{Nós Expandidos}

A tabela \ref{ne} apresenta os dados para este teste.

\begin{table}[ht]
	\centering
	\begin{tabular}{c|c|c|c|c}
		\multicolumn{1}{r|}{} & \textbf{Problema 1} & \textbf{Problema 2} & \textbf{Problema 3} & \textbf{Problema 4} \\ \hline
		\textbf{Greedy} & 10 & 21 & 5803 & 14360 \\ \hline
		\textbf{A*} & 22 & 100 & 7884 & 14680 \\ \hline
		\textbf{DFS} & 32 & 6086 & 61520 & -
	\end{tabular}
	\caption{Nós expandidos para os quatro problemas}
	\label{ne}
\end{table}

Os nós expandidos aproximam-se dos nós gerados porque a heurística pode
não ser a melhor.

\section{Nós Gerados}
A tabela \ref{ng} apresenta os dados para este teste.

\begin{table}[ht]
	\centering
	\begin{tabular}{c|c|c|c|c}
		\multicolumn{1}{r|}{} & \textbf{Problema 1} & \textbf{Problema 2} & \textbf{Problema 3} & \textbf{Problema 4} \\ \hline
		\textbf{Greedy} & 12 & 23 & 5805 & 14362 \\ \hline
		\textbf{A*} & 26 & 104 & 7888 & 14684 \\ \hline
		\textbf{DFS} & 37 & 6091 & 61525 & -
	\end{tabular}
	\caption{Nós gerados para os quatro problemas}
	\label{ng}
\end{table}

Os nós gerados são muito mais para a DFS do que para os restantes algoritmos.

\end{document} 
