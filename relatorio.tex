\documentclass[12pt,a4paper]{article}
\usepackage[utf8]{inputenc}
\usepackage[portuguese]{babel}
\usepackage{graphicx, hyperref, verbatim, multicol}

\author{Grupo 74 \and Daniel Fernandes 86400 \& Francisco Sousa 86416}
\title{Relatório 1ª Parte Projeto de IA}
\begin{document}
\maketitle

\section{Introdução}
Nesta primeira parte do projeto, foi-nos pedido um programa em Python que resolva
diferentes puzzles de uma variante do jogo Solitaire. O objectivo é encontrar uma sequência de jogadas,
independentemente do tamanho do tabuleiro e da sua configuração inicial.

Irá ser realizada uma análise para os seguintes problemas dados:

\begin{multicols}{2}
	\paragraph{Problema 1 (5x5)}
	\begin{verbatim}
		[["_","O","O","O","_"],
		["O","_","O","_","O"],
		["_","O","_","O","_"],
		["O","_","O","_","_"],
		["_","O","_","_","_"]]
	\end{verbatim}
	\paragraph{Problema 2 (4x4)}
	\begin{verbatim}
		[["O","O","O","X"],
		["O","O","O","O"],
		["O","_","O","O"],
		["O","O","O","O"]]
	\end{verbatim}
	\paragraph{Problema 3 (4x5)}
	\begin{verbatim}
		[["O","O","O","X","X"],
		["O","O","O","O","O"],
		["O","_","O","_","O"],
		["O","O","O","O","O"]]
	\end{verbatim}
	\paragraph{Problema 4 (4x6)}
	\begin{verbatim}
	[["O","O","O","X","X","X"],
	["O","_","O","O","O","O"],
	["O","O","O","O","O","O"],
	["O","O","O","O","O","O"]]
	\end{verbatim}
\end{multicols}

\section{Tempo de Execução}

\begin{table}[ht]
	\centering
	\begin{tabular}{c|c|c|c}
	\multicolumn{1}{r|}{} & \textbf{Problema 1} & \textbf{Problema 2} & \textbf{Problema 3} \\ \hline
	\textbf{DFS}          & 8987                                                                 & 6776                                                               & 45                                                             \\ \hline
	\textbf{Greedy}       & 6544                                                                 & 9876                                                               & 1456                                                           \\ \hline
	\textbf{A*}           & 879                                                                  & 87                                                                 & 23                                                            
	\end{tabular}
	\caption{Tempo de execução para os três problemas}
	\label{te}
\end{table}

\section{Nós Expandidos}

\begin{table}[ht]
	\centering
	\begin{tabular}{c|c|c|c}
	\multicolumn{1}{r|}{} & \textbf{Problema 1} & \textbf{Problema 2} & \textbf{Problema 3} \\ \hline
	\textbf{DFS}          & 8987                                                                 & 6776                                                               & 45                                                             \\ \hline
	\textbf{Greedy}       & 6544                                                                 & 9876                                                               & 1456                                                           \\ \hline
	\textbf{A*}           & 879                                                                  & 87                                                                 & 23                                                            
	\end{tabular}
	\caption{Nós expandidos para os três problemas}
	\label{ne}
\end{table}

\section{Nós Gerados}

\begin{table}[ht]
	\centering
	\begin{tabular}{c|c|c|c}
	\multicolumn{1}{r|}{} & \textbf{Problema 1} & \textbf{Problema 2} & \textbf{Problema 3} \\ \hline
	\textbf{DFS}          & 8987                                                                 & 6776                                                               & 45                                                             \\ \hline
	\textbf{Greedy}       & 6544                                                                 & 9876                                                               & 1456                                                           \\ \hline
	\textbf{A*}           & 879                                                                  & 87                                                                 & 23                                                            
	\end{tabular}
	\caption{Nós gerados para os três problemas}
	\label{ng}
\end{table}

\end{document} 
